\documentclass[a4paper]{scrartcl}
\usepackage[utf8]{inputenc}
\usepackage[english]{babel}
\usepackage{pgfplots}
\usepackage{amsmath, enumerate, amssymb, multirow, fancyhdr, color, graphicx, lastpage, listings, tikz, pdflscape, subfigure, float, polynom, hyperref, tabularx, forloop, geometry, listings, fancybox, tikz, forest, tabstackengine, cancel, multicol, algorithm}
\usepackage[noend]{algpseudocode}
\input kvmacros
\geometry{a4paper,left=3cm, right=3cm, top=3cm, bottom=3cm}
\pagestyle{fancy}
\title{Final project for the Course High Performance Programming, Philipp Noel von Bachmann}


\begin{document}
\maketitle
\section{Introduction}
    Machine Learning and Data analysis are some of the most influencial fields
    in our current society. One of the main task in Machine Learning is to give
    a prediction based on some input variables. A common algorithm to use is
    linear-least-square, which finds the best fit for a linear model. Here, we
    need an efficient way to calculate this fit. In Data analysis, a common tool
    is the Principal Component analysis, which tries to reduce the
    dimensionality of the data in a way, that the reconstruction error gets
    minimized. PCA relies on an eigenvalue decomposition. For both methods, we
    can use the QR decomposition too speed up the computation.


\section{Problem description}
    Suppose $A$ is a real, square matrix. Then it can be shown that $A$ can be
    decomposed into 
    \begin{equation}\label{eq:QR}
        A = QR
    \end{equation}
    where $Q$ is orthogonal and R is upper triangular. Finding $Q$ and $R$ is
    the task of the algorithm.

\section{Solution}
    We will implement an algorithm known as \textbf{Givens Rotation}. This
    algorithm relies on construction a sequences of matrices $G_i$, such that
    when multiplying $A$ with $G_i$, we get a new matrix with a zero at a
    predefined place. Choosing $G_i$ such that we eliminate the lower diagonal
    of A, we end up with a $R$ and by multiplying all $G_i$ with $Q$. We will
    first show how to eliminate one value at the time by constructing $G_i$ and
    then how to combine these.

    \subsection{Givens rotation}
        First, we define a Givens rotation matrix as 
        \begin{equation}\label{eq:Givens}
            G(i,j,\theta) = \begin{bmatrix} 
                1   &   &   \\
                    &   \ddots&   \\
                    &   & c_{ii} & \cdots& -s_{ij}\\
                    &   & \vdots & \ddots & \vdots\\
                    &   & s_{ji} & \cdots & c_{jj}\\
                    &   &  & & &\ddots\\
                    &   &  & & &&1\\            
                \end{bmatrix}
        \end{equation}
        where $c=\cos(\theta)$, $s=\sin(\theta)$ and any not filled out values are 0.
        Note: Therefore we can equally represent $G$ by $G(i,j,c,s)$
    
    \subsection{Eliminating one value}
        We will how to find $\theta$ to solve
        \begin{equation}
            G(i,j,\theta)^T \begin{bmatrix} 
                \times\\
                \vdots\\
                a\\
                \vdots\\
                b\\
                \vdots\\
                \times
                \end{bmatrix}
            = \begin{bmatrix} 
                \times\\
                \vdots\\
                r\\
                \vdots\\
                0\\
                \vdots\\
                \times
                \end{bmatrix}
        \end{equation}
        where $\times$ are arbitrary numbers $a,b \in \mathbb{R}$, $r=\sqrt{a^2+b^2}$. 
        A trivial solution would be 
        % \begin{align}
        %     c = \frac{a}{r} & s = \frac{-b}{r}
        % \end{align}.
        \begin{multicols}{2}
            \begin{equation}
                c = \frac{a}{r}
            \end{equation}\break
            \begin{equation}
                s = \frac{b}{r}
            \end{equation}
          \end{multicols}
        However, $r$ is prone to overflow, so we can instead store it in a
        different way. If
        \begin{itemize}
            \item $\lvert b \rvert \geq \lvert a \rvert$:\\
                $t=\frac{a}{b}$, $s=\frac{sgn(b)}{\sqrt{1+t^2}}, c=st$
            \item else:\\
                $t=\frac{b}{a}$, $c=\frac{sgn(a)}{\sqrt{1+t^2}}, s=ct$
        \end{itemize}

    \subsection{Final algorithm}
        \begin{algorithm}[H]
            \caption{Givens rotation}\label{alg:step}
            \begin{algorithmic}[1]
            \Procedure{Step}{}
            \State set $R=A$, $Q=I$
            \For{$j$ in 1 to $n$}
                \For{$i$ in $n$ down to $j+1$}
                    \State compute the Givens rotation $G(i,j,c,s)$ eliminating $R_{ij}$ with $a=i$, $b=i-1$.
                    \State set $R=G(i,j,c,s) R$, $Q = Q G(i,j,c,s)$
                    \State
                \EndFor
            \EndFor
            \EndProcedure
            \end{algorithmic}
        \end{algorithm}

        Note that we have to do it in this order, since each update affects row and row above.


\section{Random} 
    \begin{itemize}
        \item we can order them by affecting row and above
        \item best is probably to create tasks because upper rows require less work
        \item maybe need memory optimizations, but most of the operations seem to be near anyway
    \end{itemize}

    TODO:
    \begin{itemize}
        \item implement checks
        \item implement good storage of G
        \item implement efficient matmul of G
        \item implement one givens rotation
        \item implement outer loop
        \item serial optimizations
        \item parallelizations, probably with openmp
    \end{itemize}





\section{Experiments}
\section{Conclusion}
\section{References}





\end{document}

