\documentclass[a4paper]{scrartcl}
\usepackage[utf8]{inputenc}
\usepackage[english]{babel}
\usepackage{pgfplots}
\usepackage{amsmath, enumerate, amssymb, multirow, fancyhdr, color, graphicx, lastpage, listings, tikz, pdflscape, subfigure, float, polynom, hyperref, tabularx, forloop, geometry, listings, fancybox, tikz, forest, tabstackengine, cancel}
\input kvmacros
\geometry{a4paper,left=3cm, right=3cm, top=3cm, bottom=3cm}
\pagestyle{fancy}
\title{Assignment 3, Philipp Noel von Bachmann}


\begin{document}
\maketitle

\section{Problem}
    In the N-body Problem, we want to predict the movement of n bodies that
    interact with each other over time. The interaction is given by the force
    they excert on each other, which in turn changes the acceleration of each
    body, thus the velocity and position. The problem is modeled by assuming
    pairwise interaction between the bodies and taking their sum as the total
    force excerted onto each body.

    In this exercise, we use an algorithm known as the Barnes-Hut. This
    algorithm reduces the computational cost by approximating a group of starts
    that are far away as one single big star. Together with a representation
    known as a quad-tree, this enables us to reduce the computational complexity
    significantly.

\section{Solution}
    \subsection{Data structure}
        \subsubsection{Body}
            Like in the previous assignment, we have the following 4 characteristics of each body:
            \begin{itemize}
                \item velocity
                \item position
                \item mass
                \item brightness
            \end{itemize}
            Mass and brightness are represented as double, since they are both
            continuous. As position and velocity are both vectors, they are
            represented as a struct of coordinates. Since we model the problem in
            2D, each vector thus comprises of an x and y coordinate. The
            characteristics are collected in a struct called body. This format
            also allows for easy reading of the input gal file, since we can
            directly read each block into a body struct.

            In later parts \ref{sec:mem_usage}, we also keep track of the
            accelerations of each body in an array of coordinates. We opt not to
            integrate this into the body struct to be able to retain easy writing
            and reading of files.

        \subsubsection{Quadtree}
            A quadtree is a tree of nodes, where every node represents a subset
            of an area of interest. Each node can be either an internal or external node.

            An external node is a node in whose area there is only one body. If
            in the area of a node contains more than one body, it is called an
            internal node. Instead of having an body that belongs to this node,
            the node has 4 child nodes which represent the 4 subareas of this
            nodes area with their correpsonding bodies. To construct a quadtree,
            we therefore split all nodes recursively until each leaf node is an
            external node. A node whose area contains no bodies is also
            considered a external node.

            In general, we represent a node in a structure with the following attributes:
            \begin{itemize}
                \item lower: coordinate representing the lower edge of the area coverned by the node
                \item upper: coordinate representing the upper edge of the area coverned by the node
                \item children: A list containing a pointer to each of the children nodes, if there are any, else a NULL pointer.
                \item body: A pointer to the body contained by this node if an external node, else a NULL pointer.
                \item total mass: A float representing the total mass of the bodies in the area of this node.
                \item center of mass: A coordinate, representing the center of all bodies contained by this node weighted by their mass.
            \end{itemize}

    \subsection{Structure}
        The baseline code consists of the following modules/functions:
        \begin{itemize}
            \item utility functions for coordinates: compute direction and compute norm 
            \item compute force: computes the force that acts onto a body
            \item compute body update: computes the update step for each body
            \item step: handles the update step, calls the subroutines. Only this function actually changes the state of our bodys.
        \end{itemize}

        In the final code, we restructured it to the following, see mostly section \ref{sec:mem_usage} for details.
        \begin{itemize}
            \item utility functions for coordinates: compute direction and compute norm 
            \item compute acceleration: computes the acceleration that acts onto a body
            \item update body: computes the update step for each body and applies it
            \item step: calls the subroutines and iterates over all bodys
        \end{itemize}


\section{Performance and Discussion}

    \subsection{Hardware information}
        \begin{itemize}
            \item CPU: Apple M1 with 4 performance cores and 4 efficiency cores
            \item GPU: Apple M1 8 core.
            \item Compiler: Slang, Version 13.0.0
        \end{itemize}

    \subsection{Performance Measurement script}
        The performance was measured with a simple shell script to enable
        reproducability. In the script we vary the number of bodies. In the end
        we also check that the code is correct with the \textbf{compare\_gals} file. Our
        time measurement was done within the code with the help of the time
        \textbf{timeoftheday} function. We only started the time measurement
        immediatly before calling the step loop and stopped the measurement
        directly after the loop.

    \subsection{Performance and Results}

        \subsection{Calculation of $\theta_{\max}$}
        
            We calculate $\theta_{\max}$ as described in the Assignment, Section
            5.2. This results in $\theta_{\max} = 0.25$, where an $\text{\textbf{pos\_maxdiff}}
            = 0.000885759450$ was achieved.

        \subsection{Baseline results}
            The following table shows the results of the algorithm with no
            optimization for a variable number of bodies:
            \noindent\begin{tabular}{c|c|c|c}
                steps & 10 & 100 & 1000 \\
                \hline
                time (s) & 0.003 & 0.121 & 3.943\\
            \end{tabular}

            Compared to the last assignment, these times are better for the case
            of $N=10$, but worse for higher numbers.

        \subsection{Analysis of runtime}
            This section analyzes the runtime of the two main parts of the
            algorithm: Building the tree and calculating the forces.


        \section{TODO}
        ideas:
        - 
            

\section{References}
    \begin{itemize}
        \item For the timing, the \textbf{get\_wall\_seconds} from Lab06, exercise 7
        \item For data reading, the function \textbf{read\_doubles\_from\_file} from \textbf{compare\_gal\_files}, but slightly modified
    \end{itemize}



\end{document}

